% Version 2.20 of 2017/10/04
%
\documentclass[envcountsame,runningheads,orivec]{llncs}
%
\usepackage{graphicx}
% Used for displaying a sample figure. If possible, figure files should
% be included in EPS format.
%
% If you use the hyperref package, please uncomment the following line
% to display URLs in blue roman font according to Springer's eBook style:
% \renewcommand\UrlFont{\color{blue}\rmfamily}
\raggedbottom
\usepackage{sidecap}
\usepackage{microtype}
\usepackage{placeins}
\usepackage{comment}
\usepackage{url,hyperref}
\usepackage[strings]{underscore}

% language/encoding stuff
\usepackage[utf8]{inputenc}
\usepackage[english]{babel}
\usepackage[T1]{fontenc}
\usepackage{textcomp}
\usepackage[titletoc,title]{appendix}
%\usepackage[font=small,labelfont=bf]{caption}

% Maths stuff
\usepackage{latexsym,amssymb,ae,amscd,stmaryrd}
\usepackage{graphicx}% http://ctan.org/pkg/graphicx, for scalebox
\usepackage{enumerate}% http://ctan.org/pkg/enumerate
\usepackage{listings} % for code
\lstset{language=CAML}
%\lstset{basicstyle=\ttfamily\small}
\usepackage{courier}
\usepackage{mathtools}
\usepackage{array} % for \newcolumntype macro
\usepackage{ebproof} % for proof trees
%\usepackage{amsmath}
\usepackage{xtab}

\allowdisplaybreaks
\setlength{\belowdisplayskip}{1pt}

\usepackage{color}
\usepackage{xcolor}

\usepackage{wrapfig}

\usepackage{pdfpages}
%% Nikos' macros

\usepackage{paralist}
%\newcommand\nt[1]{{\color{blue}{\bf NT:} \em #1}}
\newcommand\nt[1]{{\color{blue}#1}}
\newcommand\sem[1]{\llbracket #1\rrbracket}
\newcommand\ofSig[1]{#1^\circ}
\newcommand\dom{\mathrm{dom}}
\newcommand\boldemph[1]{\textbf{\emph{#1}}}
\newcommand\doSig[1][\sigma]{\{#1\}}
\newcommand\cutout[1]{}

%These are for pretty printing 
\renewcommand{\ttdefault}{cmvtt}
%\usepackage[scaled=0.87]{helvet}
%\renewcommand{\sfdefault}{phv}


%% Yu-Yang's macros
\newcommand\TODO[1]{{\color{red}{\bf TODO:} \em #1}}

% Commands for common things
\newcommand{\ret}{ret} %  ret
\newcommand{\defval}{\mathtt{dval}} %  ret
\newcommand{\nil}{\mathtt{nil}} %  nil
\newcommand{\isnil}{\mathtt{inil}} %  nil
\newcommand{\fail}{BLA} %  fail
\newcommand{\assert}[1]{\mathtt{assert}(#1)} %  fail
\newcommand{\bigchi}{\mbox{\Large$\chi$}} % big chi
\newcommand{\refs}{\mathtt{Refs}}
\newcommand{\meths}{\mathtt{Meths}}
\newcommand{\vars}{\mathtt{Vars}}
\newcommand{\ints}{\mathtt{Int}}
\newcommand{\vals}{\mathtt{Vals}}
\newcommand{\cvals}{\mathtt{CVals}}
\newcommand{\canons}{\mathtt{CForms}}
\newcommand{\ssarefs}{\mathtt{SSAVars}}
\newcommand{\termsM}{\mathtt{Terms}}
\newcommand{\pts}{\mathtt{Pts}}
\newcommand\atype[1]{\mathtt{#1}}
\newcommand\Item[1][]{%
	\ifx\relax#1\relax  \item \else \item[#1] \fi
	\abovedisplayskip=0pt\abovedisplayshortskip=0pt~\vspace*{-\baselineskip}}
\newcommand{\preserves}{\simeq}
\newcolumntype{L}{>{$}l<{$}} % math-mode version of "l" column type
\newcolumntype{R}{>{$}r<{$}} % math-mode version of "r" column type
\newcolumntype{C}{>{$}c<{$}} % math-mode version of "c" column type
\newcommand{\llbraces}{\{\!\!\{}
\newcommand{\rrbraces}{\}\!\!\}}
\newcommand\fleq[1]{\begin{flalign*} &#1& \end{flalign*}}
\newcommand\canonsubs[1]{\llbraces #1 \rrbraces}
\newcommand\pair[1]{\langle #1 \rangle}

\newcommand{\CC}{C\nolinebreak\hspace{-.05em}\raisebox{.4ex}{\tiny\bf +}\nolinebreak\hspace{-.10em}\raisebox{.4ex}{\tiny\bf +}}
\newcommand{\CS}{C\nolinebreak\hspace{-.04em}\raisebox{.4ex}{\small\bf \#}}
\newcommand\evalbox[1]{\llparenthesis{#1}\rrparenthesis}
\newcommand{\racket}{\textsc{Racket}}
\newcommand{\horef}{\textsc{HORef}}

\newcommand{\rosette}{\textsc{Rosette}}
\newcommand{\bmctwo}{\textsc{BMC-2}}
\newcommand{\mochi}{\textsc{MoCHi}}

\lstdefinelanguage{Lambda}{%
  morekeywords={%
    if,then,else,fix,let,letrec,in,fail,nil,match,with,assert,new % keywords go here
  },%
  morekeywords={[2]int},   % types go here
  otherkeywords={:,+,-,=}, % operators go here
  literate={% replace strings with symbols
    {rarr}{{$\to$}}{1}
    {lambda}{{$\lambda$}}{1}
    {leq}{{$\leq$}}{1}
    {geq}{{$\geq$}}{1}
  },
  basicstyle={\small},
  keywordstyle={\bfseries},
  keywordstyle={[2]\itshape}, % style for types
  keepspaces,
  tabsize=2,
  columns=flexible,
  mathescape % optional
}[keywords,comments,strings]%

\newcommand{\appsigma}{\{\sigma\}}
\newcommand{\appsigmas}[1]{\{\sigma_#1\}}

%==============
% Commands for translation algorithm
\newcommand\letin[1]{\text{let}~#1~\text{in}}
% base cases:
\newcommand{\tnil}{\begin{aligned}[t]
&\llbracket M,R,C,D,\phi,\alpha,pc,\nil\rrbracket =
(\ret,(\ret = \defval) \land \phi,R,C,D,\alpha \land (pc\implies \isnil),\bot)
\end{aligned}}
\newcommand{\tfail}{NO LONGER IN USE}
\newcommand{\tval}{\begin{aligned}[t]
&\llbracket v,R,C,D,\phi,\alpha,pc,k\rrbracket = (\ret,(\ret=v)\land \phi,R,C,D,\alpha,\top)
\end{aligned}}
\newcommand{\tderef}{\begin{aligned}[t]
&\llbracket {!}r,R,C,D,\phi,\alpha,pc,k\rrbracket = (\ret,(\ret=D(r))\land \phi,R,C,D,\alpha,\top)
\end{aligned}}
\newcommand{\tlambda}{\begin{aligned}[t]
&\llbracket \lambda x.M,R,C,D,\phi,\alpha,pc,k\rrbracket =
\begin{aligned}[t]
& (\ret,(\ret=m)\land \phi,R',C,D,\alpha,\top)\\
&\text{where $R'=R[m \mapsto \lambda x.M]$ and $m$ fresh}
\end{aligned}
\end{aligned}}
\newcommand{\tpi}{\begin{aligned}[t]
		& \llbracket\pi_i\,v,R,C,D,\phi,\alpha,pc,k\rrbracket = (\ret,(\ret=\pi_i\,v)\land \phi,R,C,D,\alpha,\top)
	\end{aligned}}
\newcommand{\tassign}{\begin{aligned}[t]
& \llbracket r:=v,R,C,D,\phi,\alpha,pc,k\rrbracket =
\begin{aligned}[t]
	&\letin{C' = C[r]}\
     \letin{D' = D[r\mapsto C'(r)]} \\
	 &(\ret,((\ret={()}) \land (D'(r)=v)) \land \phi,R,C',D',\alpha,\top)
\end{aligned}
\end{aligned}}
\newcommand{\tplus}{\begin{aligned}[t]
		& \llbracket v_1 \oplus v_2,R,C,D,\phi,\alpha,pc,k\rrbracket = (\ret,(\ret=v_1\oplus v_2) \land \phi,R,C,D,\alpha,\top)
	\end{aligned}}
\newcommand{\tpair}{\begin{aligned}[t]
    	& \llbracket \pair{v_1,v_2},R,C,D,\phi,\alpha,pc,k\rrbracket = (\ret,(\ret=\pair{v_1,v_2}) \land \phi,R,C,D,\alpha,\top)
	\end{aligned}}
\newcommand{\asserts}{\begin{aligned}[t]
    	& \llbracket \assert{v},R,C,D,\phi,\alpha,pc,k\rrbracket =
         (\ret,(\ret=()) \land \phi,R,C,D,(pc \implies (v \neq 0)) \land \alpha,\top)
	\end{aligned}}
% inductive cases:
\newcommand{\tletin}{\begin{aligned}[t]
	& \llbracket \texttt{let $x=M$ in $M'$},R,C,D,\phi,\alpha,pc,k\rrbracket =\\
		&\quad \begin{aligned}[t]
			& \letin{(\ret_1,\phi_1,R_1,C_1,D_1,\alpha_1,pc_1) = \llbracket M,R,C,D,\phi,\alpha,pc,k\rrbracket}\\
			& \letin{(\ret_2,\phi_2,R_2,C_2,D_2,\alpha_2,pc_2) = \llbracket M'\{\ret_1/x\},R_1,C_1,D_1,\phi_1,\alpha_1,pc \land pc_1,k\rrbracket}\\
			&(\ret_2,\phi_2,R_2,C_2,D_2,\alpha_2,pc_1 \land pc_2)
		\end{aligned}
	\end{aligned}}
\newcommand{\tletrec}{\begin{aligned}[t]
	& \llbracket \texttt{letrec $f=\lambda x.M$ in $M'$},R,C,D,\phi,\alpha,pc,k\rrbracket =\\
		&\quad \begin{aligned}[t]
			& \letin{\text{$m$ be fresh}}\\
            & \letin{R' = R[m \mapsto \lambda x.M\{m/f\}]} \
			\llbracket M'\{m/f\},R',C,D,\phi,\alpha,pc,k\rrbracket
		\end{aligned}
	\end{aligned}}
\newcommand{\tapplym}{\begin{aligned}[t]
		& \llbracket m\,v,R,C,D,\phi,\alpha,pc,k\rrbracket =\\
		&\quad \begin{aligned}[t]
			& \letin{R(m)\text{ be }\lambda x.N} \
			\llbracket N\{v/x\},R,C,D,\phi,\alpha,pc,k-1\rrbracket
		\end{aligned}
	\end{aligned}}
\newcommand{\tifthenelse}{\begin{aligned}[t]
		& \llbracket \texttt{if $v$ then $M_1$ else $M_0$},R,C,D,\phi,\alpha,pc,k\rrbracket =\\
		&\quad \begin{aligned}[t]
			&\letin{(\ret_0,\phi_0,R_0,C_0,D_0,\alpha_0,pc_0) = 
			 \llbracket 	M_0,R,C,D,\phi,\alpha,pc \land (v = 0),k\rrbracket}\\
			&\letin{(\ret_1,\phi_1,R_1,C_1,D_1,\alpha_1,pc_1) = 
			 \llbracket M_1,R_0,C_0,D,\phi_0,\alpha_0,pc \land (v \neq 0),k\rrbracket}\\
			&\letin{C' = C_1[r_1]\cdots[r_n]\ (\varPi=\{r_1,\dots,r_n\})}\\
			&\letin{\psi_0 = (v = 0) \implies ((\ret=\ret_0) {}\land
             \bigwedge\nolimits_{r \in \varPi} (C'(r)=D_0(r)))}\\
            & \letin{\psi_1 = (v \neq 0) \implies ((\ret=\ret_1) {}\land
             \bigwedge\nolimits_{r \in \varPi} (C'(r)=D_1(r)))}\\
			& (\ret,\psi_0 \land \psi_1 \land \phi_1,R,C',C',\alpha_1,
            ((pc_0 \land (v = 0)) \lor (pc_1 \land (v \neq 0))))
		\end{aligned}
	\end{aligned}}
\newcommand{\tapplyx}{\begin{aligned}[t]
		& \llbracket x^\theta\,v,R,C,D,\phi,\alpha,pc,k\rrbracket =\\
		&\quad \begin{aligned}[t]
        &\text{if }R \upharpoonright \theta =\emptyset\text{ then } (\ret,(\ret=\defval)\land\phi,R,C,D,\alpha,\bot) \text{ else }\\
        &\letin{R \upharpoonright \theta\text{ be }\{m_1,...,m_n\} \text{ and } (R,C,\phi,\alpha) \text{ be } (R_0,C_0,\phi_0,\alpha_0)}\\
    	&\text{for each }i \in \{1,...,n\}:\\
			&\quad
			\begin{aligned}[t]
				& \letin{R(m_i)\text{ be } \lambda y_i.N}\
				\letin{(\ret_i,\phi_i,R_i,C_i,D_i,\alpha_i,pc_i) = \\
				& \qquad \llbracket N_i\{v/y_i\},R_{i-1},
				C_{i-1},D,\phi_{i-1},\alpha_{i-1},pc \land (x = m_i),k-1\rrbracket}\\
			\end{aligned}\\
			& \letin{C_n' = C_n[r_1]\cdots[r_j]\ (\varPi=\{r_1,\dots,r_j\})}\\
			& \letin{\psi = \bigwedge_{i=1}^{n}
				\left(
					(x=m_i) \implies 
					((\ret=\ret_i) \land 
					\bigwedge\nolimits_{r \in \varPi}{(C_n'(r)=D_i(r))})
\right)}\\
            & \letin{pc_n' = \bigvee_{i=1}^{n}
				\left(\vcenter{\hbox{$\displaystyle
                    pc_i \land (x = m_i)
				$}}\right)} \
			(\ret,\psi \land \phi_n,R_n,C_n',C_n',\alpha_n,pc_n')
			\end{aligned}
	\end{aligned}}
%==============

%%% Local Variables:
%%% mode: latex
%%% TeX-master: "main"
%%% End:
\begin{document}
%
\title{Pi-Calculus LTS Operational Semantics\thanks{Research funded by XXX (XXXX).}}
%
%\titlerunning{Abbreviated paper title}
% If the paper title is too long for the running head, you can set
% an abbreviated paper title here
%
\author{Yu-Yang Lin}
%
%\authorrunning{F. Author et al.}
% First names are abbreviated in the running head.
% If there are more than two authors, 'et al.' is used.
%
\institute{<INSTITUTE>}
%
\maketitle              % typeset the header of the contribution
%
\begin{abstract}

%\keywords{First keyword  \and Second keyword \and Another keyword.}
\end{abstract}
%
%
%
%\input{S-intro}

%%%%%%%%%%%%%%%%%%%%%%%%%%%%%%%%%%%%%

\section{Syntax}

\begin{align*}
\text{Labels} \quad~\alpha &::= a!a \mid a!(a) \mid a?a \mid a?(a) \mid \tau \\
\text{Prefixes} \quad~ \pi &::= \tau \mid a \pair{a} \mid a (x)\\
\text{Process Terms} \quad~ P &::= 0 \mid \pi . P \mid \nu x P \mid P + P \mid P\parallel P \mid X(\vec a)\\
\text{Process Program} \quad~ M &::= P \mid X(\vec x) = P ; M \mid \texttt{observer }a ; M
\end{align*}
where:
\begin{itemize}
\item names are ranged over $a,b,c$ and variants
\item variables are ranged over $x,y,z$ and variants
\item process terms are ranged over $P,Q$ and variants
\item prefixes are ranged over $\pi$ and variants
\item method names are ranged over $X,Y,Z$ and variants
\item programs are ranged over $M$ and variants
\end{itemize}
\[
\text{Binding Priority} \quad~ \pi.P > \nu x P > P + P > P\parallel P
\]

\section{Semantics}

\subsection{Configurations:}
\[
\texttt{global}(S,\delta,\Delta) \vdash \pair{\Gamma,L,R}
\]
where the components are as follows
\begin{itemize}
\item $\Gamma$: a set of global names
\item $L$: a set of local names and $\Gamma \cap L = \varnothing$
\item $R$: a multi-set of processes $P$ running in parallel
\item $S$: a globally accessible set of valid LTS states $(\Gamma,L,R)$ seen so far
\item $\delta$: a globally accessible set of valid LTS transitions accumulated so far
\item $\Delta$: a globally accessible immutable map of process declarations
\end{itemize}

\subsection{Notation:}
\begin{itemize}
\item $\mathcal{P}$: the set of all processes $P$
\item $\mathcal{L}$: the set of all action labels $\alpha$
\item $C_1 \to^* C_2 \xrightarrow{\alpha} \pair{\Gamma,L,R}\rhd S,\delta$: update global LTS
\[S \uplus (\Gamma,L,R) \quad \text{ and } \quad \delta \uplus (C_1,\alpha,\pair{\Gamma,L,R})\]
where $C_1$ is the last LTS-visible configuration in the current path so far
\item $\tilde{a}$: name $a$ that is globally and historically fresh at the point it is introduced
\item $\to^+$: transitive closure of $\to$
\item $\to^*$: reflexive transitive closure of $\to$
\item $C_1 \Downarrow C_2$ means $C_1$ evaluates to (converges on) $C_2$, i.e. $C_1 \to^* C_2 \not\to$
\end{itemize}

\subsection{Initialisation:}

Initialisation configuration: $\pair{M \mid \Gamma, \Delta}$

\begin{align*}
M &\xrightarrow{init} \pair{M \mid \varnothing,\varnothing}\\
\pair{X(\vec x) = P;M \mid \Gamma,\Delta} &\xrightarrow{init} 
\pair{M \mid \Gamma,\Delta[X \mapsto \pair{\vec x, P}]}
\\
\pair{\texttt{observer }a ; M \mid \Gamma,\Delta} &\xrightarrow{init} 
\pair{M \mid \Gamma \cup \{a\},\Delta}
\\
\pair{P \mid \Gamma,\Delta} 
&\xrightarrow{init} 
\texttt{global}(\{\pair{\Gamma,\varnothing,\{P\}}\},\varnothing,\Delta) \vdash \pair{\Gamma,\varnothing,\{P\}}
\end{align*}

\subsection{Semantics:}
We produce an LTS for a given program as a side-effect of building its computation tree. The computation tree is constructed as the reflexive transitive closure of the transition rules below by applying the rules according to priority and determinism. 

\subsubsection{Deterministic Structural Transitions:}
Determinism may be enforced with a queue of processes. This is strongly normalising as processes are defined finitely.
\begin{flalign*}
&\text{Flattening parallels (highest priority):}\\
&\pair{\Gamma,L,R \cup \{P \parallel Q\}}
\to_\beta
\pair{\Gamma,L,R \cup \{P, Q\}}
&\\\\
%%%%%
%=====================
%%%%%
&\text{Name allocation (high priority):}\\
&\pair{\Gamma,L,R \cup \{\nu x P\} } \to_\beta 
\pair{\Gamma,L\uplus \tilde{a},R \cup \{ P\{\tilde{a}/x\} \} }
&
\end{flalign*}
\subsubsection{Non-Deterministic Process Semantics:} Process rules have lower priority than the deterministic structural transitions. This is equivalent to fully evaluating structural $\beta$-transitions after every process transition.
\begin{flalign*}
&\text{Unobservable action:}\\
&\pair{\Gamma,L,R \cup \{\tau.P\}
} \xrightarrow{\tau}
%
\pair{\Gamma,L,R \cup \{P\}
}\rhd S,\delta\\\\
%%%%%
%=====================
%%%%%
&\text{Output to environment (fresh and known):}\\
&\pair{\Gamma \uplus a,L \uplus b,R \cup \{a \pair{b}. P\}} 
\xrightarrow{a!(b)}
%
\pair{\Gamma \uplus a \uplus b,L,R \cup\{P\}
}\rhd S,\delta&\\
%%%%%
&\pair{\Gamma \uplus a \cup b,L,R \cup \{a \pair{b}. P\} } 
\xrightarrow{a!b}
%
\pair{\Gamma \uplus a \cup b,L,R \cup\{P\}
}\rhd S,\delta&\\\\
%%%%%
%=====================
%%%%%
&\text{Input from environment (fresh and known):}\\
&\pair{\Gamma \uplus a,L,R \cup \{a (x). P\} } 
\xrightarrow{a?(\tilde{b})}
%
\pair{\Gamma \uplus a \uplus \tilde{b},L,R \cup\{P\{\tilde{b}/x\}\} 
}\rhd S,\delta \quad (\tilde{b} \not \in L)&\\
%%%%%
&\pair{\Gamma \uplus a \cup b,L,R \cup \{a (b). P\}} 
\xrightarrow{a?b}
%
\pair{\Gamma \uplus a \cup b,L,R \cup\{P\{b/x\}\}
}\rhd S,\delta&\\\\
%%%%%
%=====================
%%%%%
&\text{Internal communication:}\\
&\pair{\Gamma,L,R \cup \{a\pair{b}.P,a(x).Q
\}} \xrightarrow{\tau}
%
\pair{\Gamma,L,R \cup \{P, Q\{b/x\} \}
}\rhd S,\delta&\\\\
%%%%%
%=====================
%%%%%
&\text{Recursion:}~(N(X) = \pair{\vec{x},P})\\
&\pair{\Gamma,L,R \cup \{X \vec{a}\}
} \xrightarrow{\tau}
%
\pair{\Gamma,L,R \cup \{P\{\vec{a}/\vec{x}\}\} 
}\rhd S,\delta&\\\\
%%%%%
%=====================
%%%%%
&\text{Left of Sum:}\\
&\pair{\Gamma,L,R \cup \{P + Q\}
} \to 
%
\pair{\Gamma,L,R \cup \{P\}
}&\\\\
%%%%%
%=====================
%%%%%
&\text{Right of Sum:}\\
&\pair{\Gamma,L,R \cup \{P + Q\}
} \to 
%
\pair{\Gamma,L,R \cup \{Q\}
}&\\\\
\end{flalign*}
this  needs $S$ to contain the starting configuration.

\TODO{
\begin{itemize}
\item exhaust $\beta$ (structural) transitions before making LTS transitions
\item define some kind of normal form
\item think about defining an alpha-equivalence
\item Bruijn or Barendregt convention?
\item nominal terms?
\item final year project approach of defining an order for terms with no bound names, and renaming bound names deterministically
\end{itemize}




}

%%%%%%%%%%%%%%%%%%%%%%%%%%%%%%%%%%%%%

%
% ---- Bibliography ----
%
% BibTeX users should specify bibliography style 'splncs04'.
% References will then be sorted and formatted in the correct style.
%
%\bibliographystyle{abbrv}
%\bibliography{main}

%\input{S-biblio}
\end{document}

%%% Local Variables:
%%% mode: latex
%%% TeX-master: "main"
%%% End:
